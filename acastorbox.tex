
\documentclass[review]{elsarticle}
\usepackage{color}
\usepackage{lineno,hyperref}
\modulolinenumbers[5]

\journal{Nuclear Instruments and Methods B}

%%%%%%%%%%%%%%%%%%%%%%%
%% Elsevier bibliography styles
%%%%%%%%%%%%%%%%%%%%%%%
%% To change the style, put a % in front of the second line of the current style and
%% remove the % from the second line of the style you would like to use.
%%%%%%%%%%%%%%%%%%%%%%%

%% Numbered
%\bibliographystyle{model1-num-names}

%% Numbered without titles
%\bibliographystyle{model1a-num-names}

%% Harvard
%\bibliographystyle{model2-names.bst}\biboptions{authoryear}

%% Vancouver numbered
%\usepackage{numcompress}\bibliographystyle{model3-num-names}

%% Vancouver name/year
%\usepackage{numcompress}\bibliographystyle{model4-names}\biboptions{authoryear}

%% APA style
%\bibliographystyle{model5-names}\biboptions{authoryear}

%% AMA style
%\usepackage{numcompress}\bibliographystyle{model6-num-names}

%% `Elsevier LaTeX' style
\bibliographystyle{elsarticle-num}
%%%%%%%%%%%%%%%%%%%%%%%

\begin{document}

\begin{frontmatter}

\title{Tests of the radiation hardness of scintillators in a high energy proton-proton collider environment }


%% or include affiliations in footnotes:
\author[umd]{Joshua Kunkle\corref{mycorrespondingauthor}}
\cortext[mycorrespondingauthor]{Corresponding author}
\ead{jkunkle@cern.ch}
\author[umd]{Alberto Belloni}
\author[umd]{Jeff Calderon}
\author[rochester]{Pawel De Barbaro}
\author[umd]{Sarah C. Eno}
\author[baylor]{Kenichi Hatakeyama}
\author[umd]{Geng-Yuan Jeng}
\author[umd]{Julie Schnurr}
\author[umd]{Yao Yao}
\author[korea]{Sung Woo Youn}


\address[umd]{Dept. Physics, U. Maryland, College Park MD 30742 USA}
\address[korea]{Institute for Basic Science, Center for Axion and Precision Physics Research, IBS Center for Axion and Precision Physics Research
Room 4315, Department of Physics, Natural Science Building (E6-2), KAIST,
291 Daehak-ro, Yuseong-gu, Daejeon 305-701, South Korea}
\address[fnal]{Fermi National Accelerator Laboratory, Batavia, IL, USA}
\address[baylor]{Baylor University, Waco, Texas, USA}
\address[iowa]{The University of Iowa, Iowa City, IA, USA}
\address[rochester]{The University of Rochester, Rochester, NY, USA}

\begin{abstract}
  Radiation damage to the attenuation length and light output
  of scintillating materials may depend not just on the deposited energy, but also on the dose rate and the type and energy of the interacting particle.
  We present the
  results of measurements of the damage to several different types
  of scintillating material irradiated in the CMS collision hall
  during running with a center-of-mass eneryg of 13 TeV at the Large Hadron Collider.  The materials received a dose of {\color{red} xxx} over a person of {\color{red} xxx} months.  The light output was measured at several intermediate doses.
\end{abstract}

\begin{keyword}
organic scintillator\sep liquid scintillator\sep radiation
hardness \sep calorimetry
\end{keyword}

\end{frontmatter}

\linenumbers

\section{Introduction}
\label{sec:Introduction}
Radiation damage to the attenuation length and light output
of scintillating materials may depend not just on the deposited energy (dose),
but also on the dose rate and the type and energy of the interacting particle.
We present the
results of measurements of the damage to several different types
of scintillating material irradiated in the CMS collision hall at the Large Hadron Collider (LHC) during its operation at a center-of-mass energy
of 13 TeV during 2015.
The materials received a dose of {\color{red} xxx} over a person of {\color{red} xxx} months.  Their light output was measured at several intermediate doses.
Irradiation in the collision hall of a running high energy
proton-proton collider allows access to very low dose rates that
would not be affordable at reactors, electron linacs,
and ${\rm ^{60}Co}$ sources, with a particle type and energy
spectrum most appropriate for those designing detectors for hadron colliders.


In-situ tests are of particular interest, as several experiments
have found unexpected large radiation damage in operation
compared to expectations
based on irradiations using reactors, linacs and ${^{60}Co}$ sources.
In the CDF experiment, scintillators placed close to the
beam line received much larger damage than expected~\cite{Giokaris1993315}.
During the running of the LHC from its commissioning in 2009
through 2012, the CMS
detector was exposed to an integrated luminosity of 25 ${\rm fb^{-1}}$.  Parts of the
CMS endcap calorimeter are estimated to have received doses of 0.1 to 0.2 Mrad~\cite{ecfa2015}.
Studies of the radiation hardness of scintillator tiles
prior to installation in the detector,
using an electron linac and ${\rm ^{60}Co}$ sources,
indicated an exponential reduction in 
light output with accumulated dose, with a exponential constant of 
around 7 Mrad~\cite{vasken,ByonWagner1993263}.  
However, although the dose received by the CMS tiles was
small compared to this number,
significant light loss was observed ~\cite{phaseiitdr}.


However, experiments using scintillator at HERA saw damage
consistent with expectations.

One possible explanation is dose rate effects.

Another possible explanation is damage that is dependent on particle
type and energy.


\section{Tile designs}
\label{sec:design}

\section{Radiation parameters}
\label{sec:radiation}

\section{Measurement techniques}
\label{sec:techniques}

\section{Results}
\label{sec:Results}


\section{Conclusions}
\label{sec:Conclusions}

We presented results on radiation damage to scintillating materials
in 


\section{Acknowledgments}
The authors would like to thank Randy Ruchti of Notre Dame for
providing the capillaries.
 We would like to thank the University of Maryland
FabLab, especially {\color{red} who helped}, for help with fiber sputtering.
This work was supported in part by U.S. Department of Energy
Grant DESC0010072.

\section*{References}

\bibliography{acastorbox}

\end{document}
